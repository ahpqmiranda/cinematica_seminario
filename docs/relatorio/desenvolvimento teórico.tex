\subsection{Desenvolvimento teórico: Movimento Geral}
\label{sec:desenvolvimento}

Como comentado anteriormente, um corpo rígido pode ser submetido ao movimento generalizado em três dimensões, sendo descritos por:
uma velocidade angular \(\omega\) e uma aceleração angular \(\alpha\) em torno de um eixo de rotação, e uma velocidade linear \(v\) e uma aceleração linear \(a\) em relação a um ponto qualquer do corpo.

Se for considerado a origem do sistema de coordenadas, fora do corpo rígido, atingimos o conceito de velocidade relativa, pois a velocidade linear depende do referencial adotado.

Esta relação do movimento ocorre devido a existência de centros instanâneos de rotação que podem ser definidos por \(V_{B/A} = \omega \times r_{B/A}\) e a aceleração \(\alpha_{B/A} = \alpha \times r_{B/A} + \omega \times \left( \omega \times r_{B/A} \right)\).

Estas equações permitem que as velocidades a acelerações absolutas sejam determinadas a partir das velocidades e acelerações relativas. Ou seja:

\begin{equation}
    v_B = v_A + \omega \times r_{B/A}
\end{equation}
onde:
\begin{itemize}
    \item \(v_B\) é a velocidade do ponto B em relação ao referencial inercial;
    \item \(v_A\) é a velocidade do ponto A em relação ao referencial inercial;
    \item \(\omega\) é a velocidade angular do corpo rígido em relação ao referencial inercial;
    \item \(r_{B/A}\) é o vetor posição do ponto B em relação ao ponto A;
\end{itemize}

e também...

\begin{equation}
    a_B = a_A + \alpha \times r_{B/A} + \omega \times \left( \omega \times r_{B/A} \right)
\end{equation}
onde:
\begin{itemize}
    \item \(a_B\) é a aceleração do ponto B em relação ao referencial inercial;
    \item \(a_A\) é a aceleração do ponto A em relação ao referencial inercial;
    \item \(\alpha\) é a aceleração angular do corpo rígido em relação ao referencial inercial;
    \item \(r_{B/A}\) é o vetor posição do ponto B em relação ao ponto A;

\end{itemize}

Vale ressaltar que \(\alpha \times r_{B/A}\) são vetores e resultam em:

\begin{equation}
    \alpha \times r_{B/A} = \begin{bmatrix}
                                i        & j        & k        \\
                                \alpha_x & \alpha_y & \alpha_z \\
                                r_x      & r_y      & r_z      \\
    \end{bmatrix}
\end{equation}

e:

\begin{equation}
    \left( \omega \times r_{B/A} \right) = \begin{bmatrix}
                                               i        & j        & k        \\
                                               \omega_x & \omega_y & \omega_z \\
                                               r_x      & r_y      & r_z      \\
    \end{bmatrix}
\end{equation}
