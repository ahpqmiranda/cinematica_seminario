\thispagestyle{empty}

\begin{center}
	\begin{figure}[h]
  \centering
		\includegraphics[width=0.21\linewidth]{images/ufpa}
		\label{fig:ufpa}
	\end{figure}


	\vspace{1cm}
	\large \uppercase{UNIVERSIDADE FEDERAL DO PARÁ}\\
	\large \uppercase{INSTITUTO DE TECNOLOGIA}\\
	\vspace{7cm}
	\large \uppercase{ANÁLISE DO MOVIMENTO RELATIVO ATRAVÉS DE EIXOS EM ROTAÇÃO E TRANSLADO}\\
	\vspace{1cm}
	\large \uppercase {CINEMÁTICA DOS MECANISMOS} \\
	\vspace{7cm}
	\large {BELÉM/PA \\ 2025}

 \newpage
 \thispagestyle{empty}
 	\large\uppercase{alan henrique pereira miranda - 202102140072}\\
	\large\uppercase{Luis Felipe Sales do Carmo - 202002140040}\\
	\large\uppercase{João Victor Lima de Souza - 201702140067}\\
	\large\uppercase{João Vitor Farias e Farias  - 201902140014}\\
	\large\uppercase{Edevaldo de Jesus - 201902140098}\\
%	\large \uppercase{}\\
 \vspace{1cm}

 \singlespacing
 \hspace{8cm} % posicionando a caixa de texto
 \begin{minipage}{7cm}
	Atividade referente a disciplina de Cinemática dos Mecanismos do oitavo semestre do curso de engenharia mecânica, como parte das exigências para aprovação disciplinar. \\

	Prof. Dr.: Fábio Seturbal\\
	\vspace{1cm}

	Belém-PA 20 de fevereiro de 2025
	\vspace{4cm}
\end{minipage}

\onehalfspacing
\begin{center}

	EXAMINADOR\\
	\vspace{3cm}
	\rule{10cm}{0.15mm} \\
	Prof. Dr: Fábio Seturbal\\
	Universidade Federal do Pará - UFPA
\end{center}
\newpage

\begin{center}
    % BLOCO DE FIGURAS
    \thispagestyle{empty}
	\listoffigures
	\newpage
    % SUMARIO
    \thispagestyle{empty}
    \tableofcontents

\end{center}

\newpage
\thispagestyle{empty}

\end{center}