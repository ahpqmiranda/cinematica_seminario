\section{Common presentation elements}

\subsection{Box}

\setLayout{vertical}
\begin{frame}{Example on using box}

    \footnotesize

    \begin{ex}
        Em uma versão da linguagem BASIC, o nome de uma variável é uma sequência de um ou dois caracteres alfanuméricos, em que letras maiúsculas e minúsculas não são distinguidas. Além disso, um nome de variável deve começar com uma letra e deve ser diferente das cinco sequências de dois caracteres reservadas para o uso de comandos. Quantos nomes diferentes de variáveis são possíveis nesta versão do BASIC?
    \end{ex}

    \begin{block}{Solução}
        Pela regra da soma, $V=V_1+V_2$. Como as variáveis só podem começar com letras, temos que $V_1=26$. Pela regra do produto, há $26\cdot 36=936$ sequências de tamanho $2$ que comecem com uma letra e terminam com um caracter alfanumérico. Porém, não se deve usar $5$ variáveis reservadas. Assim, $V_2=26\cdot 36-5=931$. Logo, há $V=V_1+V_2 = 26+931=957$ nomes diferentes para variáveis nesta versão do BASIC.
    \end{block}

\end{frame}