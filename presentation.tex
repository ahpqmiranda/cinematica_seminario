\documentclass[aspectratio=169,t,xcolor=table]{beamer}
\usepackage[utf8]{inputenc}
\usepackage[brazil]{babel} %pacote que define o idioma
\usepackage{booktabs}
\usepackage{subcaption}
\usepackage{graphicx}

\usetheme{Ufg}

%-------------------------------------theorems--------------
\newtheorem{conj}{Conjetura}
\newtheorem{defi}{Definição}
\newtheorem{teo}{Teorema}
\newtheorem{lema}{Lema}
\newtheorem{prop}{Proposição}
\newtheorem{cor}{Corolário}
\newtheorem{ex}{Exemplo}
\newtheorem{exer}{Exercício}

\setbeamertemplate{theorems}[numbered]
\setbeamertemplate{caption}[numbered]

%-------------------------------------------------------------%
%----------------------- Primary Definitions -----------------%

% This command set the default Color, is also possible to choose a custom color
\setPrimaryColor{Ocean}

% First one is logo in title slide (we recommend use a horizontal image), and second one is the logo used in the remaining slides (we recommend use a square image)
\setLogos{lib/logos/itec-ufpa-logo}{lib/logos/Brasao-UFPA-preto}


% -------------------------------------- Title Slide Information
\begin{document}
    \setLayout{intro_bg} % Example of changing layout
\setBGColor{Ocean}  %Example of changing background color


\title[UFPA]{Análise de movimento relativo usando eixos transladando e rotacionando}
\subtitle{Cinemática dos Mecanismos: Atividade Final}
\author{
    Alan Miranda \inst{1}\and
    Luis do Carmo \inst{2}\and
    João Lima de Souza \inst{3}\and
    João Farias \inst{4}\and
    Edevaldo de Jesus \inst{5}\and
    Altino Dantas\inst{6}\and
}

\institute[ITEC] % (optional)
{
    Instituto de Tecnologia\\
    Universidade Federal do Pará
}
\date{2025}
%-----------------------The next statement creates the title page.
\frame[noframenumbering]{\titlepage}


%------------------------------------------------Slide 1
\setLayout{vertical} % This command define the layout. 'vertical' can be replace with 'horizontal', 'blank, 'mainpoint', 'titlepage'

\begin{frame}
    \frametitle{Sumário}
    \tableofcontents
\end{frame}
%---------------------------------------------------------

%---------------------------------------------------------Slide 2
    \include{docs/apresentacao/1 introdução}
%--------------------------------------------------------- Slide 3
      \subsection{Table}

    \begin{frame}{Example on using table}

        \begin{table}[]
            \centering
            \caption{\label{tab:1}Countries and their codes}

            \renewcommand{\arraystretch}{1.5}
            \setlength{\tabcolsep}{10pt}

            {\rowcolors{2}{}{LightGray!10}
                \begin{tabular}{ p{3cm}p{3cm}p{3cm}  }
                    \toprule
                    \textbf{Country Name} & \textbf{Code 2} & \textbf{Code 3} \\
                    \midrule
                    Afghanistan           & AF              & AFG             \\
                    Aland Islands         & AX              & ALA             \\
                    Albania               & AL              & ALB             \\
                    Algeria               & DZ              & DZA             \\
                    \bottomrule
                \end{tabular}
            }
        \end{table}

    \end{frame}
%--------------------------------------------------------- Slide 4
    \begin{frame}{Example on using image}

        \begin{figure}
            \centering
            \includegraphics[width=.9\textwidth]{lib/logos/Brasao-UFPA-com-descritivo-colorido.png}
            \caption{Template's Layouts.}
            \label{fig:layouts}
        \end{figure}

    \end{frame}
%---------------------------------------------------------


%--------------------------------------------------------- Slide 5


    \section{Changing colors and Layouts}

    \setLayout{blank} % Example of changing layout
    \setBGColor{DarkOrange}  %Example of changing background color

    \begin{frame}{Clean layout and two-column text}

        \begin{columns}

            \column{0.5\textwidth}
            This is a text in first column.
            $$E=mc^2$$
            $$ 1 + 2 + \cdots + k =  \frac{k \cdot (k + 1)}{2}.$$
            \begin{itemize}
                \item First item

                \item Second item
            \end{itemize}

            \column{0.5\textwidth}
            This text will be in the second column
            and on a second tought this is a nice looking
            layout in some cases.

            \begin{enumerate}
                \item First
                \item Second
            \end{enumerate}

        \end{columns}

    \end{frame}
%---------------------------------------------------------


%---------------------------------------------------------Slide 6
%Highlighting text
    \setLayout{vertical}
    \begin{frame}{Sample frame title}

        In this slide, some important text will be
        \alert{highlighted} because it's important. Please, don't abuse it.

        \begin{block}{Remark}
            Sample text
        \end{block}

        \begin{alertblock}{Important theorem}
            Sample text in alert box
        \end{alertblock}

        \begin{examples}
            Sample text in green box. The title of the block is ``Examples".
        \end{examples}

    \end{frame}
%---------------------------------------------------------


%---------------------------------------------------------Slide 7


    \section{Main point layout}

    \setLayout{mainpoint}
    \setBGColor{DarkPurple}
    \begin{frame}{}
        \frametitle{Preliminary Empirical Study}
    \end{frame}

 \setLayout{horizontal}
    \begin{frame}
        \frametitle{Sample frame title}
        This is a text in second frame. For the sake of showing an example.

        \begin{itemize}
            \item<1-> Text visible on slide 1
            \item<2-> Text visible on slide 2
            \begin{itemize}
                \item text subitem
            \end{itemize}
            \item<3> Text visible on slides 3
            \item<4-> Text visible on slide 4
        \end{itemize}
    \end{frame}
%---------------------------------------------------------
% Agradecimentos
    \section{Agradecimentos}
\setLayout{blank}
\begin{frame}

    \centering
    \vspace{2cm}

    \textbf{\Huge Thanks}

    \ \\

    \textbf{Doubts and Suggestions}
    \ \\

    \text{\footnotesize alan.miranda@itec.ufpa.br}

    \vspace{.5cm}
    \begin{figure}
        \centering
        \begin{subfigure}{0.2\textwidth}
            \centering
            \includegraphics[height=1.8cm]{lib/logos/Brasao-UFPA-com-descritivo-branco.png}
        \end{subfigure}%
        \qquad
        \begin{subfigure}{0.2\textwidth}
            \centering
            \includegraphics[height=2cm]{lib/logos/itec-ufpa-logo}
        \end{subfigure}

    \end{figure}

\end{frame}



\end{document}